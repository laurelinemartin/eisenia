\documentclass[12pt]{book}
\usepackage[utf8]{inputenc}
\usepackage{hyperref}
\usepackage{graphicx}

\title{Petit guide d'utilisation de Linux Mint}
\author{Association Eisenia}
\date{2021}

\begin{document}
\maketitle

\newpage
\renewcommand{\contentsname}{Table des matières}
\tableofcontents

\newpage
\renewcommand{\chaptername}{Chapitre}
\chapter{Introduction}
	\section{Introduction générale}
	\section{Introduction au logiciel libre}

\newpage
\chapter{Environnement Linux Mint}
\section{Se familiariser avec l'environnement Linux Mint}
	\subsection{Le bureau}
	\subsection{La barre des tâches}
		La barre des tâches se situe en bas de l'écran.
		Le bouton tout à gauche permet d'ouvrir le "menu démarrer" (voir section \ref{sec:menudemarrer}). 
	\subsection{Le menu démarrer}\label{sec:menudemarrer}
		Le menu démarrer se décompose en quatre parties :
		\begin{itemize}
			\item La barre latérale tout à gauche contient lesz outils de bases : le Navigateur Firefox (voir Sections \ref{sec:descfirefox} et \ref{sec:utiliserfirefox}); l'invit de commande (voir Section \ref{sec:utiliserterminal}); ...
			\item Le menu principal au centre contient toutes les catégories de logiciel
			\item Le menu de droite contient les éléments contenu dans la catégorie sélectionnée
			\item La barre de recherche en haut permet de rechercher par mots-clés 
		\end{itemize}
	\subsection{Le gestionnaire de fichiers}
		Le gestionnaire de fichier se trouve sur la barre des tâches, c'est l'icône qui est représenté par un petit dossier vert.\newline
		
	\subsection{Eteindre l'ordinateur}
		Lorsque vous avez terminer votre utilisation, il faut éteindre l'ordinateur; L'éteindre après chaque utilisation est important. Cela permet notamment de réduire la chaleur des composants, de résoudre certains bugs liés à une trop longue utilisation, etc.\newline
		Pour cela, ouvrez le menu démarrer et cliquez sur l'icône rouge en bas à gauche de la barre latérale.
		Une boîte de dialogue s'ouvre et vous propose plusieurs options.
		Cliquez sur celle toute à droite "Eteindre".
		Vorte ordinateur affiche le logo de Linux Mint puis s'éteint.

\section{Utiliser un périphérique amovible}

\section{Paramètrer le compte}
	Dans cette section, nous allons voir comment nous pouvons personnalier l'ordinateur.
	Changer le nom d'utilisateur, le mot de passe, l'avatar... peut être bien utile et plus agréable.\newline
	\subsection{Modifier le nom de l'utilisateur}
	\subsection{Modifier le mot de passe}
	\subsection{Modifier l'avatar}

\chapter{Logiciels}
\section{Les logiciels indispensables pour bien démarrer}
	\subsection{Le pack LibreOffice}
		Le pack LibreOffice est indispensable lorsque l'on souhaite rédiger des documents de toute sorte.\newline
		Il est composé de plusieurs lorgiciels.
		\begin{itemize}
			\item LibreOffice writer vous permet de rédiger des documents textes et de les mettre en forme.
			Ces documents sont au format .odt. 
			Il est aussi possible d'imprimer ses documents ou de les exporter au format .pdf via LibreOffice Writer.
			\item LibreOffice Calc vous permet de créer et d'utiliser des classeurs.
			Ces documents sont au format .ods. 
			Ils sont très utilisés pour faire des feuilles de calculs, enregistrer des données, etc.
			\item LibreOffice Impress vous permet de créer des diaporamas. 
			Ces documents sont au format .odp.
			LibreOffice Impress vous sera très utile pour préparer des supports visuels, des diaporamas pour des exposés, des réunions, etc.
			Il vous sera possible d'exporter votre document au format .pdf.
			\item LibreOffice Draw vous permet de faire toute sorte de dessins.
			Ces documents sont au format .odg.
			Très pratique pour faire un petit dessin rapide, un schéma plus complexe ou bien encore un support visuel.
			\item LibreOffice Math vous permet de rédiger des formules mathématiques.
			Ces documents sont au format .odf;
			Cet outil vous permet de prnedre en notes des formules mathématiquesdiverses, d'utiliser des opérateurs (unitaires, binaires, logiques...), des relations, de rédiger des matrices, d'utiliser des symboles, etc.
			\item LibreOffice Base vous permet de créer et de gérer une base de données relationnelles.
			Il est possible de créer une nouvelle base ou bien de se connecter à une base (JDBC par exemple). 
		\end{itemize}
	\subsection{Mozila Firefox}\label{sec:descfirefox}
		Le navigateur Mozila Firefox va vous permettre d'afficher des pages web.\newline
		Ce navigateur très connu et utilisé vous offre beacoup de paramètres d'affichage, de langue mais aussi de sécurité.
	\subsection{VLC}
	\subsection{...}

\section{Installer de nouveaux logiciels}

\newpage
\chapter{Internet}
\section{Se connecter à son réseau}
	\subsection{Avec un câble éthernet}
	\subsection{Avec le WiFi}

\section{Utiliser le navigateur Mozila Firefox}\label{sec:utiliserfirefox}

\section{Utiliser le moteur de recherche Lilo}

\newpage
\chapter{Mises à jour}
	\subsection{Pourquoi faire les mises à jour ?}
	\subsection{Effectuer les mises à jour}

\newpage
\chapter{Bonnes pratiques}
	\section{Entretien du matériel}
	\section{Sécurité sur internet}

\newpage
\chapter{Avancé}
\section{Utiliser l'invit de commande}\label{sec:utiliserterminal}
	\subsection{Présentation de l'invit de commande}
	\subsection{Les commandes de bases}


\end{document}