\documentclass[12pt]{book}
\usepackage[utf8]{inputenc}
\usepackage{hyperref}
\usepackage{graphicx}

\title{Petit guide d'utilisation de Linux Mint}
\author{Association Eisenia}
\date{2021}

\begin{document}
\maketitle

\newpage
\renewcommand{\contentsname}{Table des matières}
\tableofcontents

\newpage
\renewcommand{\chaptername}{Chapitre}
\chapter{Introduction}
	\section{Introduction générale}
	\section{Introduction au logiciel libre}

\newpage
\chapter{Environnement Linux Mint}
\section{Se familiariser avec l'environnement Linux Mint}
	\subsection{Le bureau}
	\subsection{La barre de tâche}
	\subsection{Le menu démarrer}
	\subsection{Le gestionnaire de fichiers}

\section{Utiliser un périphérique amovible}

\section{Paramètrer le compte}

\newpage
\chapter{Logiciels}
\section{Les logiciels indispensables pour bien démarrer}
	\subsection{Le pack LibreOffice}
	\subsection{Mozila Firefox}
	\subsection{VLC}
	\subsection{...}

\section{Installer de nouveaux logiciels}

\newpage
\chapter{Internet}
\section{Se connecter à son réseau}
	\subsection{Avec un câble éthernet}
	\subsection{Avec le WiFi}

\section{Utiliser le navigateur Mozila Firefox}

\section{Utiliser le moteur de recherche Lilo}

\newpage
\chapter{Mises à jour}
	\subsection{Pourquoi faire les mises à jour ?}
	\subsection{Effectuer les mises à jour}

\newpage
\chapter{Bonnes pratiques}
	\section{Entretien du matériel}
	\section{Sécurité sur internet}

\newpage
\chapter{Avancé}
\section{Utiliser l'invit de commande}
	\subsection{Présentation de l'invit de commande}
	\subsection{Les commandes de bases}


\end{document}