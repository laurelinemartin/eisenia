\documentclass[12pt]{book}
\usepackage[utf8]{inputenc}
\usepackage[T1]{fontenc}
\usepackage[french]{babel}
\usepackage{hyperref}
\usepackage{graphicx}
\usepackage{hyperref}
\usepackage{subcaption}
\usepackage{longtable}
\usepackage[table]{xcolor}
\usepackage{geometry}
\usepackage{pdfpages}

%
\setlength{\parskip}{0.15cm}


\title{Guide d'utilisation de Linux Mint}
\author{Association Eisenia}
\date{2021}

\begin{document}
\renewcommand{\contentsname}{Table des matières}
\renewcommand{\tablename}{\textsc{Tableau}}
\renewcommand{\figurename}{\textsc{Capture d'écran}}
\renewcommand{\chaptername}{Chapitre}
%\maketitle
	%\includepdf{page_de_garde/pagegarde.pdf}
\includepdf[pagecommand={\newgeometry{hmargin=0cm, vmargin=0cm}}]{page_de_garde/pagegarde.pdf}
\newpage
	\begin{center}
		\textbf{Avant-propos :}
	\end{center}\par
	Vous avez un nouvel ordinateur avec un environnement Linux Mint, mais vous ne vous y connaissez pas du tout en Linux, voir pas du tout en ordinateur ?
	Pas de panique ! 
	Ce livret reprend avec vous toutes les bases de ce système d'exploitation.\par
	Ce guide pratique se base sur le système d'expoiration Linux Mint 20 Cinnamon.
	Pas d'inquiétude, si vous disposez d'une autre version de Linux Mint, les informations contenues ici vous seront tout de même très utiles, mais les captures d'écrans seront peut-être différentes de votre système.\par
	\medskip
	Ce guide est une initiative de l'Association Eisenia dans le cadre du projet \textit{"Linux \& Populus"}.
	L'objectif principal de ce projet est de lutter contre la fracture numérique, contre les déchets électroniques, contre l'obsolescence programmée et pour la promotion de logiciels libres.
	Pour cela, du matériel informatique destiné à être jeté par les entreprises, les collectivités et les particuliers est récupéré puis reconditionné pour enfin être redistribué à un public en situation de fracture numérique.\par
	N'hésitez pas à consulter la page web\footnote{\href{https://eisenia.org/linuxetpopulus/}{https://eisenia.org/linuxetpopulus/}} et la page Facebook \footnote{\href{https://www.facebook.com/linuxpopulus/}{ttps://www.facebook.com/linuxpopulus/}} de Linux \& Populus pour en savoir d'avantage sur ce projet.
	\vfill
	\begin{center}
		\includegraphics[scale=.5]{include/linuxpop.png}
	\end{center}
	\vfill

\newpage
\tableofcontents

\chapter{Introduction}
%\section{Introduction générale}
Linux Mint est système d'exploitation.
Cela signifie qu'il fait le rôle d'interface entre l'utilisateur (vous) et la machine.
Le rôle du système d'exploitation est multiple.
Il permet aux utilisateurs de lancer des programmes, de gérer des données, des processus, etc.
Il existe beaucoup de systèmes d'exploitation différents, ici nous traiterons uniquement du système Linux Mint.
%\section{Introduction au logiciel libre}
Linux Mint est système d'exploitation libre.
C'est-à-dire, que son code est entièrement ouvert à qui veut le consulter.
Cela signifie également que tout utilisateur peut soumettre à une communauté de céveloppeurs et de développeuses une suggestion de changement dans le code du système d'exploitation.
Le libre permet également de réduire les mises à jour au stricte nécessaire.
Les données utilisées par les programmes libres sont également diminuées et ne sont pas revendues.\par
Le "libre" s'oppose au "propriétaire".
Le propriétaire siginifie que le code appartient à une entreprise privée.
Le code n'est donc pas consultable et le propriétaire est le seul décisionnaire de ce qu'il peut y mettre.\par
Même si vous ne comprenez pas ou que vous ne voulez pas vous embêter à fouiller dans, parfois, des milliers de lignes de code, le libre vous garantit une protection de vos données et une transparence sur ce qui est fait.
Le libre étant développé par une communauté, il n'y a pas d'intérêt financier à concerver et vendre vos informations.
Dans ce livret, nous explorons Linux Mint.
Nous verrons les bases de ce système d'exploitation.
Nous aborderons aussi d'autres thématiques comme l'utilisation d'Internet et les bons gestes à aborder pour garder son ordinateur en bon état le plus longtemps possible.

\chapter{Environnement Linux Mint}
\begin{minipage}[ct]{.18\textwidth}
	\centering
	\includegraphics[width=.85\textwidth]{include/lm_logo.png}
\end{minipage}
\begin{minipage}[c]{.77\textwidth}
	Linux Mint est un système d'exploitation visant à être utilisé facilement par les particuliers et les professionnels.
	C'est un système d'exploitation libre et gratuit assez récent s'appuyer sur d'autres systèmes d'exploitation de Linux (Ubuntu/Debian).
\end{minipage}\par
Le but principal de Linux Mint est de faciliter au maximum l'utilisation d'un ordinateur pour le grand public.\par
Dans ce chapitre, nous nous intéressons aux bases du système d'exploitation Linux Mint et plus particulièrement à son environnement de bureau appelé Cinnamon.
\section{Se familiariser avec l'environnement de bureau Cinnamon de Linux Mint}
	Pour bien commencer avec un nouvel envrironnement, il faut en connaître les bases.
	Savoir bien utiliser les bases de l'environnement, c'est mieux le comprendre et ainsi l'utiliser plus simplement, plus rapidement et se sentir plus à l'aise avec son matériel.\par
	Dans cette section nous allons voir les éléments basiques pour bien commencer avec ce nouvel environnement de travail.
	\subsection{Le bureau}\label{sec:bureau}
		Le bureau est le premier élément qui s'affiche sur votre écran une fois votre session ouverte.
		C'est en quelque sorte votre écran d'accueil.
		Il contient des raccourcis vers des fichiers, des dossiers ou encore des applications que vous utilisés fréquement.
		\begin{figure}[h]
			\centering
			\includegraphics[width=\textwidth]{include/bureau.png}
			\caption{Bureau Linux Mint 20}
			\label{fig:bureau}
		\end{figure}\par
		Le but du bureau est de vous faciliter les choses.
		Vous pouvez donc le personnaliser comme vous le souhaitez.
		Voyons comment nous pouvons changer le fond d'écran du bureau et comment ajouter et supprimer des éléments sur le bureau.
		\subsubsection{Changement du fond d'écran du bureau}
			Le fond d'écran du bureau est la première chose que l'on voit lorsque l'on allume l'ordinateur.
			Il est donc préférable que ce fond d'écran nous plaise.\par
			Que vous ayez une image en tête ou bien que vous souhaitiez tout simplement consulter la banque d'image de fonds d'écran proposé sur votre ordinateur, vous pouvez faire ceci :
			\begin{enumerate}
				\item Sur le bureau, faire clic droit avec la souris;
				\item Sélectionner "\texttt{Modifier l'arrière-plan du bureau}";
				\item Une sélection de fonds d'écran vous est proposé. Cliquez sur le fond d'écran qui vous intéresse et il sera immédiatement défini comme fond d'écran du bureau.
				\begin{figure}[h]
					\centering
					\includegraphics[width=.8\textwidth]{include/fondsdecran.png}
					\caption{Fenêtre pour sélectionner un nouveau fond d'écran}
					\label{fig:fondsdecran}
				\end{figure}\newline
				Sur le volet gauche de cette fenêtre, vous pouvez cliquez sur l'onglet "images" pour accéder aux images que vous avez enregistrées dans le dossier "Images" de votre ordinateur.
			\end{enumerate}
		\subsubsection{Ajout / suppression d'éléments sur le bureau}\label{sec:elementsbureau}
			Si vous utilisez très fréquement un fichier ou un logiciel et que vous ne voulez pas avoir à le chercher à chaque démarrage de votre machine, vous pouvez le mettre sur le bureau. 
			Une petite icône sera alors ajoutée au bureau en tant que "raccourcis" vers cet élément. 
			Ainsi, vous n'aurez qu'à cliquer sur cette icône pour ouvrir vos fichiers, vos dossiers ou vos applications favoris.\newline
			Pour mettre un nouvel élément sur votre bureau :
			\begin{enumerate}
				\item Ouvrir le Menu Démarrer (voir la Section \ref{menu}) et retrouver l'élément que vous voulez ajouter au Bureau
				\item Faire clic droit avec la souris sur l'élément;
				\item Sélectionner "\texttt{Ajouter au bureau}";
				\item Un raccourci a été ajouté sur votre bureau.
			\end{enumerate}\par
			Si vous ne souhaitez plus voir un élément sur votre bureau, vous pouvez le supprimer du bureau tout en le gardant sur votre ordinateur.\newline
			Attention !
			Ceci fonctionne uniquement avec les raccourcis que vous avez créés. 
			Attention ! Ne faites pas ceci avec un document que vous avez enregistré sur le bureau.
			Le supprimer le retirera définitivement de votre ordinateur.
			\begin{enumerate}
				\item Sur le bureau, faire un clic droit avec la souris sur l'élément que vous souhaitez enlever du bureau;
				\item Sélectionner "\texttt{Mettre à la corbeille}"..
			\end{enumerate}
	\subsection{La barre des tâches}\label{sec:barretaches}
		\begin{figure}[h]
			\centering
			\includegraphics[width=\textwidth]{include/barretaches.png}
			\caption{La barre des tâches}
			\label{fig:barretaches}
		\end{figure}
		La barre des tâches se situe en bas de l'écran.
		Cette barre vous permet d'accéder à des applications épinglées et aux applications en cours d'utilisation.
		Elle vous permet également d'accéder à différents gestionnaires qui ne seront pas détaillés dans cette partie.\par
		Le bouton tout à gauche permet d'ouvrir le "menu démarrer" (voir la Section \ref{sec:menu}).\par
		L'icône de la fenêtre verte vous permet d'afficher le bureau en réduisant toutes les fenêtres ouvertes.
		C'est-à-dire qu'elles ne seront plus affichées mais elles ne seront pas fermées.
		Cliquez une première fois dessus pour réduire les fenêtres et afficher le Bureau. 
		Cliquez à nouveau sur cette icône pour afficher toutes les fenêtres ouvertes.\par
		L'icône du renard blanc sur fond orange est le Navigateur Firefox qui vous permet d'accéder à Internet (voir les Sections \ref{sec:descfirefox} et \ref{sec:utiliserfirefox}).\par
		L'icône représentant un dossier vert vous permet d'accéder à vos dossiers et fichiers (voir la Section \ref{sec:fichiers}).\par
		A droite de la barre des tâches vous pouvez trouves différentes icônes telles que :
		\begin{itemize}
			\item Les rapports systèmes;
			\item Le gestionnaire de mise-à-jour;
			\item Le gestionnaire des imprimantes;
			\item Le gestionnaire des périphériques;
			\item Le gestionnaire des réseaux;
			\item Le gestionnaire du son.
		\end{itemize}\par
		Enfin tout à droite de la barre des tâches, vous trouverez l'heure, elle se synchronise automatiquement une fois l'ordinateur connécté à Internet.
		En cliquant sur l'heure, vous affichez un calandrier.
	\subsection{Le menu démarrer}\label{sec:menu}
		\begin{figure}[h]
			\centering
			\includegraphics[width=.4\textwidth]{include/menu.png}\newline
			\caption{Le Menu}
			\label{fig:menu}
		\end{figure}\par
		Le menu démarrer permet d'accéder à l'ensemble des applications qui sont présentes sur votre ordinateur.
		Comme il est ordonné, il est très simple de s'y retrouver et de l'utiliser.
		Ce menu vous sera indispensable lorsque vous aurez besoin d'un logiciel ou autre.\par
		Le menu démarrer se décompose en quatre parties :
		\begin{enumerate}
			\item La barre latérale tout à gauche contient lesz outils de bases : le Navigateur Firefox (voir Sections \ref{sec:descfirefox} et \ref{sec:utiliserfirefox}); l'invit de commande (voir Section \ref{sec:utiliserterminal}); ...;
			\item Le menu principal au centre contient toutes les catégories de logiciel;
			\item Le menu de droite contient les éléments contenu dans la catégorie sélectionnée;
			\item La barre de recherche en haut permet de rechercher par mots-clés;
		\end{enumerate}
	\subsection{Le gestionnaire de fichiers}\label{sec:fichiers}
		Le gestionnaire de fichiers vous permet de ranger et de retrouver facilement vos répertoires et vos fichiers divers.
		\begin{figure}[h]
			\centering
			\includegraphics[width=.95\textwidth]{include/fichiers.png}
			\caption{Le gestionnaire de fichiers}
			\label{fig:fichiers}
		\end{figure}\par
		Le gestionnaire de fichiers se trouve sur la barre des tâches.
		Il s'agit de l'icône qui est représenté par un petit dossier vert.
		\begin{figure}[h]
			\centering
			\includegraphics[width=\textwidth]{include/fichier_barre.png}
			\caption{Localisation du gestionnaire de fichiers sur la barre des tâches}
			\label{fig:fichiers_barre}
		\end{figure}\par
		La fenêtre qui s'ouvre vous permet d'accéder à tous les documents que vous avez enregistrés sur votre ordinateur mais aussi sur vos disques amovibles (par exemple une clé USB).
		\begin{enumerate}
			\item Chemin du répertoire courant (le répertoire que vous consultez);
			\item Ensemble des répertoires et des fichiers du répertoire courant;
			\item Accès rapide aux répertoires fréquemment utilisés
			\begin{itemize}
				\item La partie \texttt{Poste de travail} affiche l'ensemble des repertoires enregistrer sur votre disque dur interne;
				\item La partie \texttt{Réseau} affiche les répertoires partagés sur un même réseau local;
				\item La partie \texttt{Périphériques} peut apparaître lorsque un (ou plusieurs) disque(s) amovible(s) est inséré dans votre ordinateur. Le repertoire du disque apparait dans cette partie.
			\end{itemize}
		\end{enumerate}
	\subsection{Extinction de l'ordinateur}
		Lorsque vous avez fini d'utiliser l'ordinateur, il faut l'éteindre. 
		Eteindre son équipement après chaque utilisation est important. 
		Cela permet notamment de réduire la chaleur des composants, de résoudre certains bogues liés à une trop longue utilisation, de limiter sa consommation énergétique, etc.\par
		Pour éteindre votre machine :
		\begin{enumerate}
			\item Ouvrez le Menu et cliquez sur l'icône rouge en bas à gauche de la barre latérale;
%			\begin{figure}[h]
%				\centering
%				\includegraphics[width=.8\textwidth]{include/eteindre_menu.png}
%				\caption{Menu : bouton d'extinction de l'ordinateur}
%				\label{fig:eteindre_menu}
%			\end{figure}
			\item Une boîte de dialogue s'ouvre et vous propose plusieurs options;
			\begin{figure}[h]
				\centering
				\includegraphics[width=.51\textwidth]{include/eteindre.png}
				\caption{Eteindre son ordinateur}
				\label{fig:eteindre}
			\end{figure}
			\item Cliquez sur celle toute à droite "\texttt{Eteindre}" (le bouton rouge);
			\item Vorte ordinateur affiche le logo de Linux Mint puis s'éteint.
		\end{enumerate}\par
		Comme vous pouvez le voir sur la capture d'écran \ref{fig:eteindre}, vous pouvez également "\texttt{Mettre en vielle}" (votre ordinateur est toujours allumé mais consomme moins d'énergie et met les applications en pause) ou "\texttt{Redémarrer}" (l'ordinateur arrête tous les programmes et se relance) votre ordinateur.
		Mettre en veille son ordinateur peut être utile lorsque que vous prenez une pause mais que vous ne voulez pas tout arrêter par exemple.
		Redémarrer son ordinateur peut être utile lorsque celui-ci commence à ralentir ou boguer.
		Cela peut aussi être nécessaire après l'installation d'un nouveau logiciel ou d'une mise à jour.
\section{Utilisation d'un périphérique}
	Un périphérique amovible est un accessoire connecté à l'ordinateur qui vient le compléter.
	Cela peut être un clavier, une souris, une clé USB, une imprimante, une manette de jeux, etc.\par
	La plupart du temps, votre ordinateur reconnaît automatiquement votre périphérique.
	C'est le cas pour la plupart des clé USB de stockage ainsi que des disques durs externes.
	Pour d'autres accessoires commes une imprimante ou un scanner, il faut parfois installer le pilote manuellement.
	Lorsque vous brancher votre accessoire, une notification peut apparaitre en haut à droite de l'écran pour vous signifier qu'un nouveau périphérique a bien été détecté.
	S'il s'agit d'un périphérique de stockage (clé USB, disque dur externe...) une fenêtre s'ouvre automatiquement dans le répertoire du périphérique.
	S'il s'agit d'un autre périphérique, la notification peut vous indiquer qu'il faut installer le pilote adéquat.
	Cliquez sur cette notification et faites la recherche de nouveaux pilotes si cela est nécessaire.
	Installer les pilotes indiqués.
	Une fois cela fait, votre ordinateur aura peut-être besoin de redémarrer.
	Une fois redémarré, vous pourrez utiliser votre périphérique.
\section{Paramètrage du compte}
	Pour personnaliser d'avantage votre ordinateur vous pouvez modifier le nom d'utilisateur, le mot de passe, l'avatar, le thème général...
	Dans cette section, nous expliquons comment faire ces changements.
	\subsection{Modifier le nom de l'utilisateur}
		Le nom d'utilisateur est le nom que vous donner au compte enregistré sur votre ordinateur.
		Vous pouvez le personnalier, pour cela :
		\begin{enumerate}
			\item Ouvrez le Menu;
			\item Cliquez sur "Administration" puis descendez et cliquez sur "Utilisateurs et Groupes";
			\item Si besoin, entrez votre mot de passe;
			\item Cliquez sur votre compte (ici "utilisateur");
			\begin{figure}[h]
				\centering
				\includegraphics[width=\textwidth]{include/users.png}
				\caption{Liste des utilisateurs}
				\label{fig:nomuser}
			\end{figure}
			\item Cliquez sur le nom d'utilisateur actuel face au champs "Nom :"
			\begin{figure}[h]
				\centering
				\includegraphics[width=.8\textwidth]{include/nomuser.png}
				\caption{Modifier le nom d'utilisateur}
				\label{fig:nomuser}
			\end{figure}
			\item Tapez votre nouveau nom d'utilisateur puis appuyez sur la touche "\texttt{Entrée}" du clavier.
		\end{enumerate}
		Félicitations ! Vous avez changé votre nom d'utilisateur.
	\subsection{Modifier le mot de passe}
		Le mot de passe de votre ordinateur est très important. 
		Il protège votre ordinateur contre une utilisation malvaillante ou d'une erreur de manipulation en vous demandant de confirmer votre mot de passe.
		Vous pouvez le modifier, pour cela :
		\begin{enumerate}
			\item Ouvrez le Menu;
			\item Cliquez sur "Administration" puis descendez et cliquez sur "Utilisateurs et Groupes";
			\item Si besoin, entrez votre mot de passe;
			\item Cliquez sur votre compte (ici "utilisateur", voir la Capture d'écran \ref{fig:nomuser});
			\item Cliquez sur le mot de passe caché actuel face au champs "Mot de passe :"
			\begin{figure}[h]
				\centering
				\includegraphics[width=.5\textwidth]{include/mdpuser.png}
				\caption{Modifier le mot de passe}
				\label{fig:ndpuser}
			\end{figure}
			\item Tapez votre nouveau mot de passe et confirmez-le.
		\end{enumerate}
		Félicitations ! Vous avez changé votre mot de passe.
	\subsection{Modifier l'avatar}
		L'avatar de votre compte est visible lors de votre connexion à l'ordinateur et lors de l'affichage de votre compte.
		Vous pouvez le personnaliser, pour cela :
		\begin{enumerate}
			\item Ouvrez le Menu;
			\item Cliquez sur "Administration" puis descendez et cliquez sur "Utilisateurs et Groupes";
			\item Si besoin, entrez votre mot de passe;
			\item Cliquez sur votre compte (ici "utilisateur", voir \ref{fig:nomuser});
			\item Cliquez sur l'icône représentant une silhouette grise
			\begin{figure}[h]
				\centering
				\includegraphics[width=.78\textwidth]{include/avataruser.png}
				\caption{Modifier l'avatar du compte}
				\label{fig:avataruser}
			\end{figure}
			\item Sélectionnez un nouvel avatar.
		\end{enumerate}
		Félicitations ! Vous avez changé votre avatar.
	\subsection{Modifier le thème}\label{sec:theme}
		 Pour personnaliser un peu plus votre environnement de travail ou pour un meilleur confort visuel, vous aurez peut-être envie de changer le thème de votre ordinateur.
		 Pour cela, rien de plus simple :
		 \begin{enumerate}
		 	\item Ouvez le Menu;
		 	\item Cliquez sur "\texttt{Préférences}" puis descendez jusqu'à "\texttt{Thèmes}";
		 	\item Une nouvelle fenêtre s'ouvre avec les champs suivants :
		 	\begin{figure}[h]
		 		\centering
		 		\includegraphics[width=.8\textwidth]{include/themes.png}
		 		\caption{Paramétrage du thème de l'ordinateur}
		 		\label{fig:themes}
		 	\end{figure}
		 	\begin{itemize}
		 		\item \texttt{Bordure de fenêtre} : modifier la couleur d'une fenêtre;
		 		\item \texttt{Icônes} : modifier l'aspect et la couleur des icônes répertoire;
		 		\item \texttt{Contrôles} : modifier l'aspect et la couleur des boutons d'une fenêtre;
		 		\item \texttt{Pointeur de la souris} : modifier la couleur du pointeur de la souris;
		 		\item \texttt{Bureau} : modifier l'aspect et la couleur de la barre de tâches et du menu.
		 	\end{itemize}
		 \end{enumerate}

\chapter{Logiciels et applications}\label{sec:logiciels}
Sous Linux Mint, l'installation de logiciels et d'applications se fait uniquement via un portail particulier -la Logithèque- ou par lignes de commande.
La Logithèqe est le portail qui vous permet de rechercher et d'installer des applications et des logiciels directement sur votre ordinateur.\par
Dans ce chapitre, nous expliquons comment utiliser la logithèque.
Nous détaillons également quelques logiciels libres indispensables pour bien démarrer votre expérience numérique.
Seuls quelques logiciels seront détaillés dans ce chapitre, nous vous invitons à explorer la logithèque ou de consulter des sites tels que \href{https://framalibre.org/}{https://framalibre.org/} (dernière consultation le 31/03/2021) qui pourront vous inspirer et vous aider à trouver le logiciel libre de vos rêves.
\section{La Logithèque de Linux Mint}\label{sec:logitheque}
	La Logithèque présente sur votre ordinateur se présente comme une immense bibliothèques d'applications.
	Elle sera votre meilleure alliée pour rechercher et explorer rapidement et très simplement de nouveau logiciels à intégrer à votre ordinateur.\par
	Pour accéder à la Logithèque :
	\begin{enumerate}
		\item Ouvrez le Menu;
		\item Cliquez sur la section "\texttt{Administration}" puis descendez jusqu'à "\texttt{Logithèque}".
	\end{enumerate}\par
	\begin{figure}[h]
		\centering
		\includegraphics[width=.8\textwidth]{include/logitheque.png}
		\caption{Accueil de la Logithèque}
		\label{fig:logitheque}
	\end{figure}\par
	Une fois sur l'écran d'accueil, vous pouvez rechercher un logiciel de différentes manières :
	\begin{enumerate}
		\item Tapez dans la barre de recherche placée en haut à droite de la fenêtre si vous connaissez le nom du logiciel que vous chercher ou vous souhaitez effectuer une recherche par mots-clé;
		\item Découvrez un logiciel suggéré par page d'accueil;
		\item Cherchez sur la catégorie que vous souhaitez explorer.
	\end{enumerate}
	\subsection{Un exemple de recherche dans la logithèque}\label{sec:exlogitheque}
		Voici un petit exemple simple pour rechercher et installer facilement un logiciel ou une application via la logithèque.\par
		Pour cela, nous allons prendre l'exemple d'une application nous permettant d'obtenir la météo.\par
		Une fois sur l'écran d'accueil de la Logithèque, nous tapons "\textit{météo}" dans la barre de recherche située en haut à droite de la fenêtre puis nous appuyons sur la touche "\texttt{Entrée}".
%		\begin{figure}[h]
%			\centering
%			\includegraphics[width=.8\textwidth]{include/meteo1.png}
%			\caption{Recherche du mot-clé "\textit{météo}" dans la barre de recherche de la logithèque}
%			\label{fig:meteo1}
%		\end{figure}\par
		Une liste d'applications s'affiche alors. 
		Nous en choisissons une parmi cette liste et cliquons dessus.
%		\begin{figure}
%		\includegraphcis[width=.8\textwidth]{include/meteo2.png}
%		\caption{Liste de résultat pour la recherche "\textit{météo}"}
%		\label{fig:meteo2}
%		\end{figure}\par
		Le détail pour cette application s'affiche.
		On y retrouve une description, des captures d'écrans et les avis d'autres utilisateurs.
		Nous cliquons sur "\texttt{Installation}"
		\textit{faire une subfigure avec meteo4 et meteo5}
\section{Les logiciels indispensables pour bien démarrer}
Savoir télécharger des logiciels sur son ordinateur, c'est bien.
Savoir quels logiciels installer, c'est mieux !\par
Dans cette partie, nous présentons quelques logiciels simple et indispensables.
Bien d'autres logiciels seront utiles en plus de ceux présentez ici.
N'hésitez pas à explorer la logithèque.
	\subsection{Le pack LibreOffice}\label{sec:libreoffice}
		\hspace{-.8cm}
		\begin{minipage}[c]{0.3\textwidth}
			\centering
			\includegraphics[width=\textwidth]{include/lo_logo.png}
		\end{minipage}
		\begin{minipage}[c]{.65\textwidth}
			\hspace{.4cm}
			Le pack LibreOffice est indispensable lorsque l'on souhaite rédiger des documents de toute sorte.
			Il s'agit d'une "suite bureautique", c'est-à-dire d'un ensemble de logiciels vous permettant de réaliser différentes tâches de bureau.
			Grâce à une suite bureautique, vous pouvez lire et rédiger des documents, des feuilles de calculs, des diaporamas, etc.
		\end{minipage}\par
			La suite bureautique LibreOffice es très puissante et très simple à utiliser.
			Elle est composée des fonctionnalités suivantes :
		\begin{itemize}
			\item LibreOffice writer vous permet de rédiger des documents textes et de les mettre en forme.
			Ces documents sont au format .odt. 
			Il est aussi possible d'imprimer ses documents ou de les exporter au format .pdf via LibreOffice Writer.
			\item LibreOffice Calc vous permet de créer et d'utiliser des classeurs.
			Ces documents sont au format .ods. 
			Ils sont très utilisés pour faire des feuilles de calculs, enregistrer des données, etc.
			\item LibreOffice Impress vous permet de créer des diaporamas. 
			Ces documents sont au format .odp.
			LibreOffice Impress vous sera très utile pour préparer des supports visuels, des diaporamas pour des exposés, des réunions, etc.
			Il vous sera possible d'exporter votre document au format .pdf.
			\item LibreOffice Draw vous permet de faire toute sorte de dessins.
			Ces documents sont au format .odg.
			Très pratique pour faire un petit dessin rapide, un schéma plus complexe ou bien encore un support visuel.
			\item LibreOffice Math vous permet de rédiger des formules mathématiques.
			Ces documents sont au format .odf;
			Cet outil vous permet de prnedre en notes des formules mathématiquesdiverses, d'utiliser des opérateurs (unitaires, binaires, logiques...), des relations, de rédiger des matrices, d'utiliser des symboles, etc.
			\item LibreOffice Base vous permet de créer et de gérer une base de données relationnelles.
			Il est possible de créer une nouvelle base ou bien de se connecter à une base (JDBC par exemple). 
		\end{itemize}\par
		La suite bureautique de LibreOffice vous permet donc de lire et de rédiger des fichiers sous des formats de documents dits ouverts (.odf) mais aussi les formats de la suite bureautique de Microsoft (.docx, xlsx...).\par
		Pour en savoir plus sur le pack LibreOffice et sur ses nombreuses fonctionnalités, vous pouvez consulter le site officiel \footnote{
		\href{https://fr.libreoffice.org}{https://fr.libreoffice.org} \textit{(dernière consultation le 02/04/2021)}}.
	\subsection{Mozilla Firefox}\label{sec:descfirefox}
		\hspace{-.8cm}
		\begin{minipage}[c]{.7\textwidth}\par
			\hspace{.4cm}
			Mozilla Firefox ext un navigateur web.
			C'est-à-dire que son rôle est d'afficher des pages web (par exemple des pages au format .html ou .php).
			Comme son nom l'indique, un navigateur vous permet de "naviguer" entre différentes pages web via des liens.
			Mozilla Firefox est très connu et très utilisé à travers le monde.
			Il vous offre beaucoup de paramètres d'affichages, de langues mais aussi de sécurité.
			C'est également un navigateur libre.
		\end{minipage}
		\begin{minipage}[c]{.3\textwidth}
			\centering
			\includegraphics[width=\textwidth]{include/mf_logo.jpg}
		\end{minipage}\par
		Pour un navigateur web, cela garanti notamment une certaine sécurité sur nos données personnelles.
		En effet, Firefox n'a aucun intérêt à revendre les données de ces utilisateurs, donc les données collectées ne servent qu'au bon fonctionnement du logiciel.
		Ce navigateur permet également de bloquer les traqueurs publicitaires.\par
		N'hésitez pas à consulter le site officiel de Mozilla Firefox\footnote{\href{https://www.mozilla.org/fr/firefox/features/}{https://www.mozilla.org/fr/firefox/features/} (dernière consultation le 06/04/2021)} pour plus d'informations.
		Vous pouvez aussi consulter la liste des extensions\footnote{\href{https://addons.mozilla.org/fr/firefox/}{https://addons.mozilla.org/fr/firefox/} (dernière consultation le 06/04/20201)} proposées par Mozilla Firefox.
		Les extensions (ou plugins) sont des "petits plus" qu'on ajoute au navigateur pour l'améliorer et pour obtenir une fonctionnalité bien particulière.

\chapter{Internet}
Maintenant que nous nous sommes familiarisés avec cet environnement, il est temps d'aller surfer sur le web.
Internet, c'est la porte ouverte d'une connaissance mondiale partagée.
S'il permet d'accéder à une multitude d'informations et de contenus divers, il est important de rappeler qu'Internet est aussi un endroit où des personnes malvaillantes peuvent se trouver.
Elles peuvent propager de fausses informations (fake news en anglais), recourir à des virus ou à des logiciels malvaillants pour entrer dans votre ordinateur, etc.
Du contenu se trouvant sur Internet peut aussi ne pas convenir aux enfants.
Il faut donc s'en protéger et adopter les bons gestes lorsque l'on utilise internet et garder un esprit critique face à ce qu'on peut y trouver\par
Dans ce chapitre consacré à Internet, nous verrons dans un premier temps comment nous connecter à Internet.
Dans un deuxième temps, nous étudirons le navigateur Mozilla Firefox qui sera notre "porte" pour aller sur le web.
Dans un troisième temps, nous allons voir comment utiliser le moteur de recherche Lilo qui nous permettra de faire nos premières recherches sur internet.
Enfin, nous aborderons la question de la sécurité sur Internet.
\section{Se connecter à son réseau}\label{sec:connexion}
	Avant toute chose, nous devons être connecté à Internet. 
	C'est-à-dire que nous devons établir un lien entre l'ordinateur et le réseau.
	Pour cela, deux façons de procéder :
	\begin{enumerate}
		\item en se connectant via un un câble reliant l'ordinateur à la box ;
		\item en se connectant via un le WiFi.
	\end{enumerate}
	\subsection{Avec un câble éthernet}
		Pour connecter votre ordinateur, vous devez utiliser un câble RJ45 (éthernet).
		L'embout ressemble à ceci :
		\begin{figure}[h]
			\centering
			\includegraphics[scale=0.045]{include/rj45.jpg}
			\label{fig:rj45}
			\caption{Câble éthernet RJ45}
		\end{figure}\par
		Utiliser ce câble en le branchant d'un côté à votre ordinateur et de l'autre à votre box (ou dans la prise éthernet de votre mur si elle existe).
		Une fois branché des deux côtés, l'icône "Internet" à droite sur la barre des tâches doit avoir changé et ressembler à ceci :
		\begin{figure}[h]
		\begin{subfigure}{\textwidth}
			\centering
			\includegraphics[width=\textwidth]{include/connexion_barre.png}
			\caption{Icône indicant l'état de connexion sur la barre des tâches}
			\label{fig:barre_taches_connexion}
		\end{subfigure}\newline
		\begin{subfigure}{.5\textwidth}
		  \centering
		  \includegraphics{include/non_connect.png}
		  \caption{Avant le branchement du \newline câble RJ45}
		  \label{fig:non_connect_rj45}
		\end{subfigure}
		\begin{subfigure}{.5\textwidth}
		  \centering
		  \includegraphics{include/connect.png}
		  \caption{Après le branchement du \newline câble RJ45}
		  \label{fig:connect_rj45}
		\end{subfigure}
		\caption{Affichage de l'état de connexion avec un câble éthernet}
		\label{fig:connexion_rj45}
		\end{figure}\par
		Une notification a pu aussi apparaître tout en haut à droite de votre écran.
		Elle doit ressember à ceci :
		\begin{figure}[h]
			\centering
			\begin{subfigure}{.49\textwidth}
				\centering
				\includegraphics[width=\textwidth]{include/notif_deco.png}
				\caption{Notification "ordinateur déconnecté" d'Internet}
				\label{fig:notif_deco_rj45}
			\end{subfigure}
			\begin{subfigure}{.5\textwidth}
				\centering
				\includegraphics[width=\textwidth]{include/notif_co.png}
				\caption{Notification "ordinateur connecté" à Internet}
				\label{fig:notif_co_rj45}
			\end{subfigure}
			\label{fig:notif_connexion_rj45}
			\caption{Notifications de connexion au réseau Internet}
		\end{figure}\par
		Vous voilà maintenant connecter à Internet !
	\subsection{Avec le WiFi}
		Avant de commencer, si vous disposez d'un ordinateur portable, votre ordinateur dispose probablement d'un système de connexion au WiFi intégré et vous pouvez directement essayer de vous connecter.
		Si vous disposez d'un ordinateur de bureau, vous aurez probablement besoin d'un dispositif de connexion WiFi externe.
		Nous vous conseillons d'utiliser une clé USB WiFi.
		Elles sont très simple d'utilisation.
		Pour la plupart de ces clés WiFi, il suffit de les brancher dans l'ordinateur et vous pourrez vous connecter aux réseaux WiFi.
		\begin{figure}[h]
			\centering
			\includegraphics[width=.22\textwidth]{include/clewifi.jpg}
			\caption{Exemple d'une clé USB WiFi}
			\label{fig:clewifi}
		\end{figure}\par
		Pour vous connecter, cliquez sur l'icône indiquant l'état de connexion de votre ordinateur. Cette icône se situe à droite sur la barre des tâches.
		\begin{figure}[h]
			\centering
			\includegraphics[width=\textwidth]{include/connexion_barre.png}
			\caption{Icône indicant l'état de connexion sur la barre des tâches}
			\label{fig:barre_connexion}
		\end{figure}\par
		Un petit menu s'affiche alors.
		Vérifier que le bouton face à "Wi-Fi" est bien en position marche.
		Cliquez sur le réseau WiFi qui vous intéresse.
		Une boîte de dialogue s'ouvre pour vous demander un mot de passe.
		Entrer le mot de passe de votre réseau et cliquez sur "Se conecter".
		\begin{figure}[h]
			\centering
			\includegraphics[width=.51\textwidth]{include/wifi_mdp.png}
			\caption{Boîte de dialogue de connexion à un réseau WiFi}
			\label{fig:connexion_wifi}
		\end{figure}
		Si le mot de passe est incorrect, alors la boîte de dialogue vous demandera à nouveau d'entrer le mot de passe.\par
		La boîte de dialogue disparait et l'icône "Internet" à droite sur la barre des tâches doit avoir changé et ressembler à ceci :
		\begin{figure}[h]
			\centering
			\includegraphics{include/connect_wifi.png}
			\label{fig:iconeconnecte}
			\caption{Icône WiFi "connecté"}
		\end{figure}
		\newline
		Une notification a pu aussi apparaître tout en haut à droite de votre écran.
		Elle doit ressember à ceci :
		\begin{figure}[h]
		\centering
			\begin{subfigure}{.51\textwidth}
				\centering
				\includegraphics[width=\textwidth]{include/notif_deco.png}
				\caption{Notification "ordinateur déconnecté" d'Internet}
				\label{fig:notif_deco_wifi}
			\end{subfigure}
			\begin{subfigure}{.48\textwidth}
				\centering
				\includegraphics[width=\textwidth]{include/notif_wifi.png}
				\caption{Notification "ordinateur connecté" à Internet}
				\label{fig:notif_co_wifi}
			\end{subfigure}
			\caption{Notifications de connexion à un réseau WiFi}
			\label{fig:notif_wifi}
		\end{figure}\par
		Félicitations, vous êtes connecté à Internet !
\section{Utiliser le navigateur Mozilla Firefox}\label{sec:utiliserfirefox}
	Nous avons vu dans la Section \ref{sec:deffirefox} de nombreuses informations relatives à ce navigateur libre.
	Si ce n'est pas déjà fait, nous vous encourageons à lire cette section pour prendre connaissance des fonctionnalités présentées par Firefox.\par
	Ici, nous allons voir comment utiliser le navigateur Mozilla Firefox.\par
	Pour commencer, ouvrez le navigateur web.
	Pour cela, vous pouvez cliquez sur l'icône "Firefox" qui se trouve sur votre barre des tâche à droite de l'îcône du menu démarrer .
	\begin{figure}[h]
		\centering
		\includegraphics[width=\textwidth]{include/mf_barre.png}
		\caption{Localisation du navigateur web Mozilla Firefox sur la barre des tâches}
		\label{fig:mf_barre}
	\end{figure}\par
	Une nouvelle fenêtre s'ouvre et affiche votre page d'accueil.
	Ici, notre page d'accueil est le moteur de recherche Lilo (voir la Section \ref{sec:ilo}).
	\begin{figure}[h]
		\centering
		\includegraphics[width=.9\textwidth]{include/accueil_mf.png}
		\caption{Ouverture de Mozilla Firefox}
		\label{fig:accueil_mf}
	\end{figure}
	\begin{enumerate}
		\item Les onglets.\newline
				Vous pouvez ouvrir plusieurs pages dans une seule fênetre de navigation.
				Ceci vous permet de vous retrouver plus facilement dans votre espace.
				Pour ouvrir un nouvel onglet, cliquez sur le "+";
		\item Ce bouton vous permet de revenir à la page précédente.;
		\item Ce bouton vous permet de revenir à la page suivante (si vous avex cliquez sur le bouton \texttt{2});
		\item Ce bouton vous permet de "rafraîchir la page.
				La page web en cours d'utilisation se recharge et se relance;
		\item Ce bouton vous permet de revenir à l'écran d'accueil;
		\item Cette barre affiche l'adresse URL sur laquelle vous vous trouvez actuellement.
				Cette barre vous permet également d'effectuer une recherche en entrant un ou des mots-clé ou directement une adresse URL.
				Cliquer sur l'étoile à droite de cette barre vous permet de "marquer la page en cours".
				Cela peut vous être utile lorsque qu'il s'agit d'une page que vous consultez souvent.
				Il vous suffit de vous rendre dans vos marque-pages pour la retrouver au lieu d'effectuer une recherche à chaque fois.
		\item Ce bouton vous permet d'afficher votre historique des pages web que vous avez parcouru mais aussi vos marque-pages, vos téléchargements, vos captures d'écrans, etc.;
		\item Ce bouton vous permet d'afficher tous vos marque-pages;
		\item Ce bouton vous permet d'accéder à différents paramètres du navigateur.
	\end{enumerate}
\section{Utiliser le moteur de recherche Lilo}\label{sec:lilo}
	Pour effectuer vos recherches sur internet, vous aurez besoin d'un moteur de recherche.
	Un moteur de recherche permet, grâce à des mots-clés, de faire des recherches sur le web.
	A partir de ces mots-clés, le moteur de recherche récupère les pages web les plus pertinentes et en faire une liste.
	Le moteur de recherche est donc un allié indispensable pour effectuer vos recherches.\par
	Dans cette section, nous vous présentons le moteur de recherches Lilo.
	\subsection{Présentation de Lilo}
		\hspace{-.8cm}
		\begin{minipage}[c]{.38\textwidth}
			\centering
			\includegraphics[width=.9\textwidth]{include/lilo_logo.png}
		\end{minipage}
		\begin{minipage}[c]{.62\textwidth}
		\hspace*{.4cm}
		Lilo est un moteur de recherche solidaire.
		Il permet de soutenir des projets à caractères sociaux et/ou environnementaux.
		En faisant vos recherches, vous pouvez choisir d'attribuer des "gouttes d'eau" à un ou plusieurs projets solidaires de votre choix.
		Ces gouttes d'eau sont utilisées pour aider à financer
		\end{minipage}
		ces projets.
		Plus vous faites de recherches, plus des annonceurs s'affichent dans vos résultats.
		À chaque apparition, les annonceurs reversent quelques centimes à Lilo.
		cet argent est reversé au projet que vous avez sélectionné.\par
		Lilo n'accepte aucun traqueur.
		Lorsque vous faites une recherche, Lilo ne transmet pas ces données à des comtenus publicitaires ou autres.\par
		Lilo concerve les données personnelles privées.
		Les données personnelles que vous transmettez à Lilo (données de compte, adresse IP, etc.) sont chiffrées et utilisées dans l'unique but de faire fonctionner correctement le moteur de recherche.\par
		Lilo se présente de la manière suivante :
		\begin{figure}[h]
			\begin{subfigure}{.5\textwidth}
				\centering
				\includegraphics[width=\textwidth]{include/lilo.png}
				\caption{Lilo - La page de recherche}
			\end{subfigure}
			\begin{subfigure}{.5\textwidth}
				\centering
				\includegraphics[width=\textwidth]{include/lilo_menu.png}
				\caption{Lilo - Le menu}
			\end{subfigure}
			\caption{Le moteur de recherche Lilo}
			\label{fig:lilo}
		\end{figure}
		\begin{enumerate}
			\item La barre de rechercher.
					Tapez les mots-clés pour votre recherche puis appuyer sur la touche \texttt{Entrée};
			\item Le compteur de gouttes d'eau.
					Visualisez combien de gouttes d'eau vous avez collecté depuis que vous utilisés Lilo;
			\item Le compte Lilo.
					Inscrivez-vous / connectez-vous à votre compte Lilo.
					Ce compte est facultatif;
					Pas besoin d'avoir un compte Lilo pour faire des recherches.
			\item L'accueil.
					Revenez sur la page d'accueil de recherche de Lilo;
			\item Le menu :\newline
			\begin{enumerate}
				\item \texttt{Accueil} vous permet de revenir à la page de recherche de Lilo.
				\item \texttt{Soutenir un projet} vous permet de sélectionner le (ou les) projet(s) de votre choix en donner vos gouttes d'eau.
				\item \texttt{Proposer un projet} permet à des structures de soumettre leur pojet solidaire à Lilo afin que les internautes puissent leur donner des gouttes d'eau.
				\item \texttt{Vie privée} renvoie vers un descriptif de la gestion des données personnelle par Lilo.
				\item \texttt{FAQ} renvoie vers une page de Questions/Réponses fréquentes.
				\item \texttt{Lilo} renvoie vers une page décrivant le fonctionnement de Lilo.
				\item \texttt{Personnaliser} vous permet d'effectuer des réglages par rapport au moteur de recherche.
				\item \texttt{Mon compte} vous permet de vous connecter ou de vous créer un compte Lilo.
				\item \texttt{Use Lilo in English} vous permet de mettre Lilo en anglais.
				\item Ce compteur affiche le nombre d'euros collectés et reverser par tous les utilisateurs à des projets solidaires.
			\end{enumerate}
		\end{enumerate}
		Si vous le désirez vous pouvez afficher des options dans votre page d'accueil Lilo.
		Elles apparaîtront sous la barre de recherche (\texttt{1}).
		Vous pouvez ajouter à votre page d'accueil :
		\begin{itemize}
			\item Vos favoris.
					Accédez plus rapidement à des pages web en les ajoutant à vos favoris LIlo;
			\item Le fil d'actualité.
					En descendant dans votre page d'accueil, vous pouvez consulter les actualités.
		\end{itemize}	
	\subsection{Effectuer sa première recherche sur Internet}
		Pour effectuer une recherche, rien de plus simple ! 
		Rendez-vous sur votre moteur de recherche, nous prendrons comme exemple Lilo.
		Une fois sur la page d'accueil de Lilo, tapez les mots-clés pour votre recherche.
		Par exemple "Pastèque".
		\begin{figure}[h]
			\centering
			\includegraphics[scale=0.3]{include/pasteque1.png}
			\label{fig:pasteque1}
			\caption{Recherche du mot "Pastèque" dans Lilo}
		\end{figure}
		Appuyer en suite sur la touche \texttt{Entrée} de votre clavier.\par
		Et voilà, le moteur de recherche vous affiche une liste de page web en lien avec le mot clé "Pastèque".
		\begin{figure}[h]
			\centering
			\includegraphics[scale=0.3]{include/pasteque2.png}
			\label{fig:pasteque2}
			\caption{Liste des résultats pour le mot "Pastèque" dans Lilo}
		\end{figure}

\chapter{Mises à jour}\label{sec:maj}
	Dans ce chapitre, nouvs abordons le point des mises à jour.
	Vous vous demandez peut-être à quoi servent les mises à jours, et pourquoi il est parfois nécessaire de les effectuer. 
	C'est ce que nous expliquons dans la première partie de ce chapitre.
	Nous verrons ensuite comment effectuer des mises à jour avec l'environnement Linux Mint.
	\section{Pourquoi faire les mises à jour ?}\label{sec:maj_pourquoi}
		De manière globale, les mises à jour permettent à une application, un logiciel, un système, d'ajouter de nouvelles fonctionnalités, de corriger certaines erreurs ou bien de réparer certaines failles de sécurité.
		Elles sont faites par les développeurs pour être appliquées simplement sur le terminal (ordinateur, téléphone, etc.) de tous les utilisateurs.
		Les mises à jour se présentent sous la forme d'une nouvelle version à télécharger.
		Donc, une mise à jour permet d'améliorer l'ensemble du logiciel.
		D'une part, elle peut l'améliorer sur le plan de l'utilisation (nouvelles fonctionnalités, simplification de l'utilisation, correction de bogues, etc.).
		Et d'autre part, elle peut l'améliorer sur le plan de la sécurité et résolvant certaines failles.\par
		Dans le cas des systèmes d'exploitation de Linux, les mises à jour sont développées par la communauté du système.
		C'est-à-dire, que ce sont des personnes non-rémunérées, passionnées et soucieuses d'améliorer Linux pour les utilisateurs de Linux (dont ils font partis).
		Les mises à jour sont donc peu fréquentes pour ces systèmes d'exploitation.
		Quand une nouvelle mise à jour est déployée (i.e. téléchargeable par tous les utilisateurs) elles sont souvent synônymes de corrections de bogues ou de suppression de faiblesses dans la sécurité.
		Les mises à jour avec Linux sont travaillées plus longuement par la communauté et elles proposent des versions stables des systèmes (versions avec peu de bogues).
		Par conséquent, il est nécessaire de vérifier régulièrement si des mises à jour sont disponibles.
		Il est recommandé de vérifier si des mises à jour sont disponible tous les deux mois environ.
	\section{Effectuer les mises à jour}
		Comment faire une mise à jour sur un ordinateur Linux Mint ? 
		Notez qu'il faut que vous soyez connectés à Internet tout le temps de la recherche et de l'installation des mises à jour.
		Sans cela, les dernières mises à jour écritent par les développeurs ne peuvent pas vous parvenir. 
		Votre ordinateur ne sait même pas qu'une nouvelle mise à jour est disponible.
		Si vous ne savez pas comment vous connecter à Internet, vous pouvez consulter la Section \ref{sec:connexion}.\par
		Pour rechercher de nouvelles mises à jour :
		\begin{enumerate}
			\item Ouvrez le gestionnaire des mises à jour. Cliquez sur l'icône en forme de bouclier sur la barre des tâches, à droite;
			\begin{figure}[h]
				\centering
				\includegraphics[width=\textwidth]{include/maj_barre.png}
				\caption{Localisation de l'icône du gestionnaire des mises à jour sur la barre des tâches}
				\label{fig:maj_barre}
			\end{figure}
			\item Une fenêtre s'ouvre et peut vous afficher différentes informations :
			\item Une fenêtre de bienvenue peut s'afficher la première fois.
					Vous pouvez simplement lire et cliquer sur "\texttt{Valider}";
			\begin{figure}[h]
				\centering
				\includegraphics[width=.4\textwidth]{include/maj_accueil.png}
				\caption{Page de bienvenue du gestionnaire des mises à jour}
				\label{fig:maj_accueil}
			\end{figure}
				\begin{figure}[h]
					\begin{subfigure}{.5\textwidth}
						\centering
						\includegraphics[width=\textwidth]{include/maj_ajour.png}
						\caption{Aucune mise à jour à effectuer}
						\label{fig:maj_ajour}
					\end{subfigure}
					\begin{subfigure}{.5\textwidth}
						\centering
						\includegraphics[width=\textwidth]{include/maj_afaire.png}
						\caption{Liste des mises à jour à effectuer}
						\label{fig:maj_afaire}
					\end{subfigure}
					\caption{Gestionnaire des mises à jour}
					\label{fig:maj}
				\end{figure}
			\begin{itemize}
				\item Si aucune mise à jour n'est à faire, il est indiqué que "\texttt{\textbf{Votre système est à jour}}".
				Vous pouvez fermer la fenêtre (Capture d'écran (voir Capture d'écran \ref{fig:maj_ajour});
				\item Si des mises à jour sont requises, alors la liste de ces mises à jour est affichée (voir Capture d'écran \ref{fig:maj_afaire}).
			\end{itemize}
			\item Si rien n'est coché, cliquez sur "\texttt{Tout sélectionner}" ou cocher uniquement les mises à jour que vous souhaitez effectuer (voir Capture d'écran \ref{fig:maj_select});
			\begin{figure}[h]
				\centering
				\includegraphics[width=.6\textwidth]{include/maj_select.png}
				\caption{Sélectionner toutes les mises à jour}
				\label{fig:maj_select}
			\end{figure}
			\item Cliquez sur le bouton "\texttt{Installer les mises à jour}".
					Votre mot de passe pourra vous être demandé, entrez-le et valider (voir Capture d'écran \ref{fig:maj_install});
			\begin{figure}[h]
				\centering
				\includegraphics[width=.6\textwidth]{include/maj_install.png}
				\caption{Installer les mises à jours}
				\label{fig:maj_install}
			\end{figure}
			\item Les mises à jour peuvent prendre du temps, ne fermez pas la fenêtre de chargement pendant ce temps.
		\end{enumerate}
		Félicitation ! Votre système et vos logiciels sont à jour !

\chapter{Utilisation du Terminal}\label{sec:terminal}%{Avancé}
Maintenant que vous vous êtes bien familiarisés avec l'environnement de Linux Mint, que diriez-vous d'explorer des fonctions plus avancées et très pratiques que propose Linux ?\par
%Dans ce chapitre, nous vous présentez des fonctionnalités de Linux qui permettent de faire plus de choses sur son ordinateur et ce, de manière plus rapide.
%Nous présentons notamment le Terminal.
%\section{Utiliser le Terminal}\label{sec:terminal}
Le Terminal permet d'entrer des mots-clé capables d'être compris et exécuté par un ordinateur.
Il est bien souvent déroutant, voir effrayant pour les moins adèptes d'informatique d'entre nous.
Pourtant, son utilisation peut s'avérer bien plus rapide voir indispensable dans certains cas.
Il faut noter que lorsqu'on a un problème avec notre ordinateur sous un système d'exploitation Linux et que l'on cherche de l'aide sur Internet, une solution en ligne de commande est très souvent donnée par les internautes.
Dans ce cas, rien de compliqué, il suffit de copier et de coller la ligne de commande indiquée et d'appuyer sur la touche "\texttt{Entrée}" de votre clavier pour résoudre le prolème.\par
Dans ce chapitre, nous démystifions donc le Terminal et nous présentons quelques commandes de bases pour mieux l'apprivoiser.
\section{Présentation du Terminal}
	Le Terminal est un programme qui vous permet d'entrer des commandes (voir la Section \ref{sec:commandes}) destinées à être exécutées par votre ordinateur.
	Ces commandes vous permettent d'intéragir directement (sans passer par un programme tiers) avec votre machine.
	Elles vous seront utiles dans toutes les situations : naviguer / interagir avec les répertoires et les fichiers, interagir avec le système d'exploitation, faire des mises à jour, télécharger des logiciels, etc.\par
	Pour imager, voyez le Terminal comme une communication ouverte directe entre vous et votre ordinateur.
	\begin{figure}[h]
		\centering
		\includegraphics[width=\textwidth]{include/terminal_barre.png}
		\caption{Localisation de l'icône du Terminal sur la barre des tâches}
		\label{fig:terminal_barre}
	\end{figure}
	\begin{figure}[h]
		\centering
		\includegraphics[width=.8\textwidth]{include/terminal.png}
		\caption{Une fenêtre du Terminal de Linux Mint 20}
		\label{fig:terminal_ouvert}
	\end{figure}\par
	Quand on ouvre le Terminal, une pemière ligne s'affiche à l'écran.
	Elle est du type :
	\begin{center}
		\vspace*{-.4cm}
		\texttt{utilisateur@ThinkCentre-A70:$\sim$ \$ }
	\end{center}
	Cette ligne vous donne plusieurs informations :
	\begin{itemize}
		\item \texttt{utilisateur} correspond à l'identifiant de la session;
		\item \texttt{@} est un séparateur;
		\item \texttt{ThinkCentre-A70} est le nom de la machine;
		\item \texttt{$\sim$} indique le chemin courant. $\sim$ est l'abréviation pour \texttt{/home/<IDENTIFIANT\_MACHINE>};
		\item \texttt{\$} marque la fin du chemin courant;
		\item Le rectangle blanc à la fin est le curseur;
	\end{itemize}\par
	Désormais, vous savez à quoi sert et à quoi ressemble le Terminal. 
	Vous allez pouvoir entrer vos premières commandes et tester ses fonctionnalités de base.
\section{Commandes de bases}\label{sec:commandes}
	Une commande dans le Terminal Linux se présente sous la forme :
	\begin{center}
		\texttt{nom\_de\_la\_commande [OPTION(S)] <ARGUMENT(S)>}
	\end{center}\par
	Le nom de la commande est très souvent en minuscule.
	Il est possible d'entrer une (ou plusieurs) option pour affiner la commande.
	Un argument est "l'objet" sur lequel vous souhaitez appliquer la commande. 
	Il peut y a voir plusieurs arguments si la commande l'exisge.
	Dans les exemples du tableau \ref{tab:commande}, une nommanclature particulière est utilisée :
	\begin{itemize}
		\item Pour indiquer qu'il s'agit d'un mot-clé strict à entrer dans le Terminal tel quel, il sera écrit en minuscule;
		\item Pour indiquer qu'il s'agit d'un mot-clé non obligatoire à sélectionner parmi une liste définie (une option), il sera écrit entre crochets \texttt{[]} et en majuscule;
		\item Pour indiquer qu'il s'agit d'un mot-clé qui change selon la situation (un argument, par exemple, le nom d'un fichier), il sera écrit entre chevrons \texttt{<>} et en majuscule.
	\end{itemize}\par
	L'espace démarque la fin du nom de la commande et le début d'une option ou d'un argument.
	Un espace est donc très important pour la bonne exécution de la commande.
	Si le répertoire ou le fichier dans lequel vous souhaitez travailler contient un espace, nous vous conseillons de le renommer en amont pour vous faciliter la saisie de commandes.
	Vous pouvez aussi ajouter devant les espaces dans le nom du répertoire un \texttt{\textbackslash }.
	Ainsi, lors de l'exécution de la commande, l'ordinateur comprendra, grâce au \textbackslash que le nom du répertoire ou du fichier et composé d'un espace et qu'il ne s'agit pas du passage à une autre partie de la commande.
	Par exeple, si votre répertoire s'appelle "\texttt{Hello World}", vous devez entrer en ligne de commande "\texttt{Hello\textbackslash~World}".\par
	Ci-dessous, nous vous présentons un tableau \ref{tab:commande}.
	Il contient quelques commandes de bases à connaître pour utiliser plus efficacecement le Terminal.
	N'hésitez pas à taper ces commandes dans votre Terminal, pour observer leur action.\par
	La liste des commandes présentée ici est non-exhaustive, bien d'autres commandes existent.
	N'hésiter à rechercher sur Internet des commandes Terminal.
	Vous pouvez aussi consulter le manuel du Terminal en tapant :
	\begin{center}
		\texttt{man <NOM\_D'UNE\_COMMANDE>}.
	\end{center}
	Par exemple : \texttt{man man} ou \texttt{man sudo} ou \texttt{man mkdir}, etc.
	\begin{longtable}{|p{.3\textwidth}|p{.4\textwidth}|p{.3\textwidth}|}\hline
		\rowcolor{lightgray}Commande & Descripion & Exepmle\\\hline
		\endhead
		\texttt{cd <CHEMIN>} & Se rendre dans le répertoire passé en argument & \texttt{cd Documents/}\\\hline
		\texttt{pwd} & Affiche le chemin absolu du réperoire courant & \texttt{pwd}\\\hline
		\texttt{ls [OPTION]} & Afficher le contenu du répertoire courant & ls\\\hline
		\texttt{mkdir [OPTION] <NOM\_DU\_DOSSIER>} & Permet de créer un nouveau dossier à l'emplacement courant & \texttt{mkdir toto}\\\hline
		\texttt{cp [OPTION] [CHEMIN/]SOURCE [CHEMIN/]DESTINATION} & Copier le document \texttt{SOURCE} dans le répertoire \texttt{DESTINATION} & \texttt{cp image.jpg Images/Photos/}\\\hline
		\texttt{mv [OPTION] [CHEMIN/]SOURCE [CHEMIN/]DESTINATION} & Déplacer le document \texttt{SOURCE} dans le répertoire \texttt{DESTINATION} & \texttt{cp Images/image2.jpg Images/Autres/}\\\hline
		\texttt{rm [OPTION] <NOM>} & Supprimer un fichier & \texttt{rm helloworld.txt}\\\hline
		\texttt{rmdir <NOM\_REPERTOIRE>} & Supprimer un répertoire vide & \texttt{rmdir toto}\\\hline
		\texttt{sudo <...>} & Utiliser les droits Super-Utilisateur sur une commande. Attention ! Utiliser les privilèges de super-utilisateur implique d'enlever les "gardes-fous" que le système met en place pour éviter de faire une action irréversible. Il faut donc être sûr de la commande et de son utilité & \texttt{sudo rm hello.txt}\\\hline
		\texttt{cat <NOM\_DU\_FICHIER>} & Afficher le contenu d'un fichier. Attention ! Certains fichiers utilisent des codes particuliers qui peuvent être mal retranscrits dans le terminal & \texttt{cat toto.txt}\\\hline
		\texttt{cat [OPTION] <FICHIER1> <FICHIER2> > <DESTINATION} & Mettre à la suite les textes des fichiers \texttt{FICHIER1} et \texttt{FICHIER2} dans le fichier \texttt{DESTINATION}. Si ce dernier fichier existe déjà, son texte est remplacé, sinon le fichier est créé. & \texttt{cat hello.txt toto.txt > hello\_toto.txt}\\\hline
		\texttt{file [OPTION] <NOM\_DU\_FICHIER} & Afficher les méta-données d'un fichier & \texttt{file toto.txt}\\\hline
		\texttt{locate [OPTION] <MOT(S)>} & Liste tous les fichiers dont le nom contient le ou les mots passés en arguments. S'il y a plusieurs mots, mettre un * entre chaque mot. & locate toto*tata\\\hline
		\texttt{zip [OPTION] <NOM\_ARCHIVE>.zip <NOM>} & Compresser un fichier ou un répertoire. & zip toto.zip toto/\\\hline
		\texttt{unzip [OPTION] <NOM\_ARCHIVE>} & Décompresser une archive & \texttt{unzip toto.zip}\\\hline
		\texttt{apt update} & Mettre à jour la liste des paquets (i.e. logiciels). Cela consiste à rechercher de nouvelles versions. & \texttt{sudo apt update}\\\hline
		\texttt{apt upgrade} & Mettre à jour les paquets  installés sur l'ordinateur (à effectuer après la commande \texttt{apt update}) & \texttt{sudo apt upgrade}\\\hline
		\texttt{apt autoremove} & Supprimer les paquets devenus obselètes (\ à effectuer après la commande \texttt{apt upgrade}). & \texttt{sudo apt upgrade}\\\hline
		\texttt{apt install [OPTION] <NOM\_DU\_PAQUET} & Installer de nouveaux paquets sur votres ordinateur. Souvent, vous aurez besoin de rechercher sur internet le nom du paquet exact à installer & \texttt{sudo apt install pacman}\\\hline
		\texttt{./<NOM>} & Exécute un programme & \texttt{./pacman}\\\hline
	\caption{Liste de commandes de bases}
	\label{tab:commande}
	\end{longtable}\par
	Pour terminer ou annuler une commande vous pouvez appuyer sur \texttt{CTRL+Z}.

\chapter{Bonnes pratiques}
	Un ordinateur est un objet très fragile. 
	Il est crucial d'en prendre soin et d'être délicat avec lui.
	Mais pas seulement. 
	Un ordinateur est aussi un outil très puissant, Internet également, on peut y trouver toutes sortes d'informations mais aussi laisser beaucoup d'informations à notre sujet sans le vouloir.\par
	Dans ce chapitre, nous allons voir les bons gestes à avoir avec son matériel informatique.
	nous allons aussi voir les bons réflexes à adopter en utilisant Internet.
	\section{Entretien du matériel}
		%Entretenir votre matériel régulièrement est très important. 
		%Cela permettra d'éviter beaucoup de problèmes grâce à de simples gestes.
		\subsection{La poussière}
			Votre ordinateur devient très vite un nid à poussière.
			En effet, lorsqu'il s'agit d'une tour fixe, nous avons tendance à la laisse dans un coin, sous un bureau et on ne la bouge plus.
			Un ordinateur portable n'est pas à l'abri de la poussière non plus.
			Tous les ordinateurs sont exposés aux miettes de nourriture, aux poils d'animaux, aux cheveux, etc. Et tout cela s'accumule dans l'ordinateur ! 
			Pouvant entrainer, d'une part, un bruit très désagréable.
			Et d'autre part, à des dysfonctionnements et à un ralentissement de votre système.
			Il faut donc le dépoussiérer régulièrement.
			Environ une fois tous les deux ou trois an.\par
			Pour cela, munissez-vous de votre aspirateur ou de votre sèche-cheveux sur le mode souffle froid.
			Ne surtout pas utiliser un souffle chaud !
			Cela endommagerait les circuits de votre ordinateur.
			Si vous disposez d'un compresseur d'air, c'est encore mieux.
			Mettez-vous de préférence en extérieur car beaucoup de poussière va sortir de votre ordinateur.
			Si vous ne pouvez pas vous mettre en extérieur, ouvez les fenêtres.
			Eteignez et débranchez votre ordinateur et ouvrez-le.
			En prenant garde à ne pas abîmer les composants de votre ordinateur, soufflez (ou aspirez) la poussière présente.
			Vous pouvez vous aider d'un chiffon pour attraper la poussière dans les recoins.
		\subsection{La chaleur}
			La chaleur est un des ennemis de votre ordinateur.
			En effet, lorsque le système chauffe trop, cela peut entrainer des bogues, des erreurs dûes à des modifications électroniques sous l'effet de la chaleur.
			Mais aussi, à trop haute température, les composants peuvent commencer à fondre.
			C'est d'ailleurs pour éviter ces soucis qu'un (voir plusieurs) système de refroidissement est installé dans les ordinateurs. 
			Une bonne pratique est de ne pas laisser votre ordinateur proche d'une source de chaleur (chaffage, cheminée, soleil,...). 
			Pensez aussi à éteindre régulièrement votre ordinateur.
			Les composants à l'intérieur chauffe beaucoup lorsqu'ils sont en activité.
		\subsection{La condensation}
			Voici un autre ennemi de votre ordinateur : la condensation.
			La condensation, c'est le passage de l'eau de l'état de vapeur vers l'état liquide. En d'autres termes, c'est lorsque de la vapeur d'eau se refroidie pour devenir de petites goutelettes liquides.
			Dans un ordinateur, la condensation est très dangeureuse.
			Elle peut endommager définitivement les composants de votre appareil.
			L'eau qui s'infiltre dans votre ordinateur peut occasionner un court-circuit, oxyder les composants de votre matériel, voir les griller.\par
			Il est donc important de préserver votre matériel informatique de toute pièce trop humide, mal isolée et mal chauffée (salle de bain, cuisine, etc.).\par
			Même avec ces précautions, il n'est pas impossible que vous versiez malencontreusement un liquide (un verre d'eau par exemple) sur votre ordinateur.
			Dans ce cas, il faut agir vite pour essayer d'éviter de trop grands dégâts.
			Tout d'abord, débranchez votre ordinateur de la prise (et retirer la batterie si possible).
			Eteignez votre ordinateur.
			Cela évitera le court-circuit.
			Puis débranchez tous les périphériques amovibles pour éviter de les endommager.
			Retirez le plus d'eau possible sur votre ordinateur en utiliser un linge sec et absorbant.
			Placez votre ordinateur de manière à faciliter l'égouttement.
			N'utiliser pas de séche-cheveux ou d'autres appareils de ce type ! 
			Cela pourrait chauffer vos composants et repousser l'eau sur d'autres composants et les endommager.
	\section{Sécurité sur internet}
		Internet est une porte ouverte sur beaucoup de connaissances partagées entre tous les utilisateurs.
		Cet outil donne accès à beaucoup d'informations et ce, en très peu de temps.\par
		Mais si Internet est si puissant, il reste néanmoins dangereux pour votre ordinateur ainsi que pour vos données personnelles.
		Des personnes malvaillantes peuvent glisser des virus dans des programmes que vous téléchargez.
		Ces virus peuvent être de toutes sortes. Ils peuvent :
		\begin{itemize}
			\item Lire ce que vous faites sur votre ordinateurs; 
			\item Ouvrir des "portes dérobées" pour extraire des données de votre ordinateur;
			\item Altérer des composants logiques ou physiques de votre ordinateur;
			\item Bloquer votre ordinateur et vous demander une somme d'argent pour récupérer vos données.
		\end{itemize}
		Il faut donc se protéger de ces virus informatiques.
		Pour cela, il faut tout d'abord utiliser un anti-virus, celui-ci effectuera périodiquement des "scans" de votre ordinateur pour vérifier qu'aucun logiciel malvaillant s'y est glissé.
		Evitez de télécharger des fichiers de sources inconnues.
		Sur Linux Mint, pour télécharger un nouveau logiciel, il faut passer par la Logithèque, ce qui assure une plus grande sécurité face aux virus.
		Cepandant, vous serez amenés à ouvrir et à télécharger des documents provenant d'autres utilisateurs.
		Assurez-vous de toujours connaître votre interlocuteur (par mail par exemple). 
		Si vous ne connaissez pas la personne qui vous envoie un document, ne l'ouvrez pas.
	\section{Gestion du stockage}
		Votre ordinateur a un espace de stockage limité.
		C'est-à-dire qu'il peut enregistrer en mémoire qu'une certaine quantité de contenus.
		Lorsque votre espace de stockage arrive à saturation, vous ne pouvez plus enregistrer de nouveaux éléments sur votre ordinateur.
		Mais, la saturation de votre espace de stockage engendre également des problèmes internes à votre ordinateur.
		Il peut manquer de place en mémoire pour effectuer des opérations et causer un arrêt de système.
		Donc, il faut toujours un minimum d'espace libre (surtout quand on utiliser un SSD) pour permettre le bon fonctionnement de votre ordinateur.
		\subsection{Suppression d'éléments plus utilisés}
			Une bonne pratique est de supprimer régulièrement les logiciels et les documents que vous n'utilisez plus sur votre ordinateur.
			Cela fera de la place sur votre disque de stockage facilement.
		\subsection{Déplacement de documents vers des périphériques externes}
			Une autre bonne pratique est de déplacer vos documents (textes, photos, vidéos, etc.) vers des périphériques externes tels que des clés USB de stockages ou des disques durs externes.
			Vous pouvez également transférer vos fichiers sur des outils web appelés "Cloud".
			Il s'agit d'un service sur internet qui vous permet d'enregistrer sur des machines de stockages (i.e. des serveurs) vos documents.
			Il existe une multitude d'entreprises qui proposent un service de Cloud en ligne tel que IONOS\footnote{\href{https://www.ionos.fr/solutions-bureau/hidrive-stockage-en-ligne}{https://www.ionos.fr/solutions-bureau/hidrive-stockage-en-ligne} (dernière consultation le 21/04/2021)} et bien d'autres.
			Vos fichiers seront enregistrés sur des serveurs appartenant à l'entreprise qui gère ce Cloud.
			Nous vous recommandons de sauvegarder vos données sur vos propres disques de stockages externes.
			Cela vous permettra de les conserver dans un endroit plus sûr et de mieux contrôler le bon état de vos disques (contrairement à des serveurs distants).
			Il faut aussi noter que le Cloud à l'avantage de vous permettre d'accéder à vos documents depuis n'importe quel appareil et ce à tout moment.\par
			Sauvegarder ses documents sur un support externe est une bonne pratique qui constitue un double avantage :
			\begin{itemize}
				\item D'une part, vous faites de la place sur votre disque de stockage interne tout en conservant vos fichiers;
				\item Et d'autre part, vous conserverez vos documents même si votre ordinateur vient à boguer.
				En effet, vous n'êtes pas à l'abri d'une défaillance grave de votre système ou tout simplement d'un choc qui engendrerait de graves problèmes.
				Conserver vos fichiers ailleurs que sur votre ordinateur vous évitera de perdre l'ensemble de vos travaux et de vos souvenirs.
			\end{itemize}

\end{document}